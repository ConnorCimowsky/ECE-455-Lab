\documentclass[11pt]{article}

\usepackage{
    courier,
    listings,
    pdfpages,
    float
}

\usepackage{geometry}

\lstset{
    basicstyle=\ttfamily,
    breaklines=true,
    aboveskip=20pt,
    showstringspaces=false
}

\title{Connor Cimowsky\protect\\20427800}
\date{}

\begin{document}

\maketitle

Below is the definition of the \texttt{init\_timer} function I used for
configuring the hardware timer. Since its frequency is 25 MHz, I scaled it by
25\,000 in order to increment the \texttt{TC} register at a rate of 1000 Hz.
This means that the value in \texttt{MR0} corresponds to the desired period of
timer interrupts, in milliseconds. When the bottom 2 bits of \texttt{MCR} are
set to 1, an interrupt will be generated when the value in the \texttt{TC}
register reaches the value stored in \texttt{MR0}, and then the \texttt{TC}
register will be reset. To start the timer, I set the \texttt{TCR}'s reset and
enable bits accordingly in \texttt{start\_timer}.

\begin{lstlisting}[language=c, frame=single]
void init_timer(int period_ms)
{
    LPC_TIM0->PR = 25000;
    LPC_TIM0->MR0 = period_ms;
    LPC_TIM0->MCR = 0x03;
}
\end{lstlisting}
\vspace{1em}

To enable interrupts for the \texttt{INT0} button, I enabled general-purpose
I/O for P2.10 and subsequently configured it for input. Then, I set the
appropriate bit of \texttt{IO2IntEnF} to 1 so that interrupts would be fired on
the falling edge of the \texttt{INT0} signal. Finally, I enabled interrupts for
\texttt{EINT3} on the interrupt controller.

\begin{lstlisting}[language=c, frame=single]
void init_int0(void)
{
    LPC_PINCON->PINSEL4 &= ~(3 << 20);
    LPC_GPIO2->FIODIR   &= ~(1 << 10);

    LPC_GPIOINT->IO2IntEnF |= (1 << 10);
    NVIC_EnableIRQ(EINT3_IRQn);
}
\end{lstlisting}
\vspace{1em}

To configure the LEDs, I set the appropriate GPIO direction bits for
output and cleared the output pins.

\begin{lstlisting}[language=c, frame=single]
void init_led(void)
{
    LPC_GPIO1->FIODIR |= 0xb0000000;
    LPC_GPIO2->FIODIR |= 0x0000007c;

    LPC_GPIO1->FIOCLR |= ~(0x0);
    LPC_GPIO2->FIOCLR |= ~(0x0);
}
\end{lstlisting}
\vspace{0.6em}

To implement strict interrupt scheduling, I disabled \texttt{EINT3} interrupts
and fired a one-shot timer whenever the \texttt{INT0} button was pressed:

\begin{lstlisting}[language=c, frame=single]
void EINT3_IRQHandler(void)
{
    LPC_GPIOINT->IO2IntClr |= (1 << 10);

    NVIC_DisableIRQ(EINT3_IRQn);

    flash_message();
    start_timer();
}
\end{lstlisting}
\vspace{0.6em}

When the timer interrupt is received after 5 seconds, I re-enable
\texttt{EINT3} interrupts and disable timer interrupts:

\begin{lstlisting}[language=c, frame=single]
void TIMER0_IRQHandler(void)
{
    LPC_TIM0->IR = 0x01; // clear interrupt

    NVIC_DisableIRQ(TIMER0_IRQn);
    NVIC_EnableIRQ(EINT3_IRQn);
} 
\end{lstlisting}
\vspace{0.6em}

For the bursty scheduler, I followed similar steps, except I only disabled
\texttt{EINT3} interrupts when a counter variable reached a value of 3.
Additionally, I used an asynchronous timer to periodically reset this counter
and re-enable \texttt{EINT3} interrupts every 10 seconds.

\end{document}
