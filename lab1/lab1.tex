\documentclass[11pt]{article}

\usepackage{
    courier,
    listings
}

\usepackage{geometry}

\lstset{
    basicstyle=\ttfamily,
    breaklines=true,
    aboveskip=20pt,
    showstringspaces=false,
    emph={uint32_t},
    emphstyle={\bfseries}
}

\title{Connor Cimowsky\protect\\20427800}
\date{}

\begin{document}

\maketitle

Below is my implementation of the timing function. From experimentation in the
lab, I determined that the outer loop should iterate \texttt{10*delay\_ms}
times. To fine-tune the timing, I allowed the timer to run for 10 minutes and
then scaled the inner loop accordingly. The value I arrived at is 1536.

\begin{lstlisting}[language=c, frame=single]
void sleep(uint32_t delay_ms)
{
    int i, j;
    int noop_counter;

    __disable_irq();

    for (i = 0; i < (10 * delay_ms); i++) {
        for (j = 0; j < 1536; j++) {
            noop_counter += 1;
        }
    }

    __enable_irq();
}
\end{lstlisting}

\vspace{1em}

After running the timer for 10 minutes, the software approach yielded a time of
09:58, while the hardware approach yielded a time of 10:00. Although the
software approach could have been further calibrated to exactly 10:00, it is
still not fully predictable. For reasons including variations in clock speed,
branch prediction, and caching, the nested loops may not always take the same
amount of time to execute. This means that even on the same CPU, the software
timer's accuracy can't be guaranteed, even after calibration. However, the
hardware timer can be configured by simply setting the prescale and match
control registers. The result is a timer that will fire interrupts at the
correct interval, regardless of the CPU (ignoring negligible variations due to
interrupt latency, etc.).

\end{document}
